%%
%% Automatically generated file from DocOnce source
%% (https://github.com/hplgit/doconce/)
%%
%%


%-------------------- begin preamble ----------------------

\documentclass[%
oneside,                 % oneside: electronic viewing, twoside: printing
final,                   % draft: marks overfull hboxes, figures with paths
10pt]{article}

\listfiles               %  print all files needed to compile this document

\usepackage{relsize,makeidx,color,setspace,amsmath,amsfonts,amssymb}
\usepackage[table]{xcolor}
\usepackage{bm,ltablex,microtype}

\usepackage[pdftex]{graphicx}

\usepackage{fancyvrb} % packages needed for verbatim environments
\usepackage{minted}
\usemintedstyle{default}

\usepackage[T1]{fontenc}
%\usepackage[latin1]{inputenc}
\usepackage{ucs}
\usepackage[utf8x]{inputenc}

\usepackage{lmodern}         % Latin Modern fonts derived from Computer Modern

% Hyperlinks in PDF:
\definecolor{linkcolor}{rgb}{0,0,0.4}
\usepackage{hyperref}
\hypersetup{
    breaklinks=true,
    colorlinks=true,
    linkcolor=linkcolor,
    urlcolor=linkcolor,
    citecolor=black,
    filecolor=black,
    %filecolor=blue,
    pdfmenubar=true,
    pdftoolbar=true,
    bookmarksdepth=3   % Uncomment (and tweak) for PDF bookmarks with more levels than the TOC
    }
%\hyperbaseurl{}   % hyperlinks are relative to this root

\setcounter{tocdepth}{2}  % levels in table of contents

% prevent orhpans and widows
\clubpenalty = 10000
\widowpenalty = 10000

% --- end of standard preamble for documents ---


% insert custom LaTeX commands...

\raggedbottom
\makeindex
\usepackage[totoc]{idxlayout}   % for index in the toc
\usepackage[nottoc]{tocbibind}  % for references/bibliography in the toc

%-------------------- end preamble ----------------------

\begin{document}

% matching end for #ifdef PREAMBLE

\newcommand{\exercisesection}[1]{\subsection*{#1}}


% ------------------- main content ----------------------



% ----------------- title -------------------------

\thispagestyle{empty}

\begin{center}
{\LARGE\bf
\begin{spacing}{1.25}
PHY321: Classical Mechanics 1
\end{spacing}
}
\end{center}

% ----------------- author(s) -------------------------

\begin{center}
{\bf Homework 3, due Monday February 8${}^{}$} \\ [0mm]
\end{center}

\begin{center}
% List of all institutions:
\end{center}
    
% ----------------- end author(s) -------------------------

% --- begin date ---
\begin{center}
Feb 6, 2021
\end{center}
% --- end date ---

\vspace{1cm}


\paragraph{Practicalities about  homeworks and projects.}
\begin{enumerate}
\item You can work in groups (optimal groups are often 2-3 people) or by yourself. If you work as a group you can hand in one answer only if you wish. \textbf{Remember to write your name(s)}!

\item Homeworks are available ten days  before the deadline. 

\item How do I(we)  hand in?  You can hand in the paper and pencil exercises as a  scanned document. For this homework this applies to exercises 1-5. Alternatively, you can hand in everything (if you are ok with typing mathematical formulae using say Latex) as a jupyter notebook at D2L. The numerical exercise(s) (exercise 6 here) should always be handed in as a jupyter notebook by the deadline at D2L. 
\end{enumerate}

\noindent
\paragraph{Introduction to homework 3.}
This week's sets of classical pen and paper and computational
exercises deal with the motion of different objects under the
influence of various forces. The relevant reading background is
\begin{enumerate}
\item chapter 2 of Taylor (there are many good examples there) and

\item chapters 5-7 of Malthe-Sørenssen.
\end{enumerate}

\noindent
In both textbooks there are many nice worked out
examples. Malthe-Sørenssen's text contains also several coding
examples you may find useful.

There are several pedagogical aims we have in mind with these exercises:
\begin{enumerate}
\item Get practice in setting up and analyzing a physical problem, finding the forces and the relevant equations to solve;

\item Analyze the results and ask yourself whether they make sense or not;

\item Finding analytical solutions to problems if possible and compare these with numerical results. This teaches us also how to understand errors in numerical calculations;

\item Being able to solve (in mechanics these are the most common types of equations) numerically ordinary differential equations and compare the solutions where possible with analytical solutions;

\item Getting used to studying physical problems using all possible tools, from paper and pencil to numerical solutions;

\item Then analyze the results and ask yourself whether they make sense or not.
\end{enumerate}

\noindent
The above steps outline important elements of our understanding of the
scientific method. Furthermore, there are also explicit coding skills
we aim at such as setting up arrays, solving differential equations
numerically and plotting your results.  Coding practice is also an
important aspect. The more we practice the better we get (hopefully).
From a numerical mathematics point of view, we will solve the differential
equations using Euler's method (forward Euler).

The code we will develop can be reused as a basis for coming homeworks. We can
also extend the numerical solver we write here to include other methods (later) like
the modified Euler method (Euler-Cromer, midpoint Euler) and more
advanced methods like the family of Runge-Kutta methods and the Velocity-Verlet method.

At the end of this course, we will thus have developed a larger code
(or set of codes) which will allow us to study different numerical
methods (integration and differential equations) as well as being able
to study different physical systems. Combined with analytical skills,
the hope is that this can allow us to explore interesting and
realistic physics problems. By doing so, the hope is that can lead to
deeper insights about the laws of motion which govern a system.

And hopefully you can reuse many of the above solvers in other courses (our ideal).

Enough talk!  Here we go and best wishes.

\paragraph{Exercise 1 (20 pt), Electron moving into an electric field.}
An electron is sent through a varying electrical
field. Initially, the electron is moving in the $x$-direction with a velocity
$v_x = 100$ m/s. The electron enters the field when it passes the origin. The field
varies with time, causing an acceleration of the electron that varies in time
\[
\bm{a}(t)=\left(−20 \mathrm{m/s}^2 −10\mathrm{m/s}^3t\right) \bm{e}_y
\]
\begin{itemize}
\item 1a (4pt) Find the velocity as a function of time for the electron.

\item 1b (4pt)  Find the position as a function of time for the electron.
\end{itemize}

\noindent
The field is only acting inside a box of length $L = 2m$.
\begin{itemize}
\item 1c (4pt)  How long time is the electron inside the field?

\item 1d (4pt)  What is the displacement in the $y$-direction when the electron leaves the box. (We call this the deflection of the electron).

\item 1e (4pt)  Find the angle the velocity vector forms with the horizontal axis as the electron leaves the box.
\end{itemize}

\noindent
\paragraph{Exercise 2 (10 pt), Drag force.}
Taylor exercise 2.3

\paragraph{Exercise 3 (10 pt), Falling object.}
Taylor exercise 2.6

\paragraph{Exercise 4 (10 pt), and then a cyclist.}
Taylor exercise 2.26


\paragraph{Exercise 5 (10 pt), back to a falling ball and preparing for the numerical exercise.}
\textbf{Useful material: Malthe-Sørenssen chapter 7.5 and Taylor chapter 2.4.}

In this example we study the motion of an object subject to a constant force, a velocity dependent
force. We will  reuse the code we develop here in homework 4 for a position-dependent force.

Here we limit ourselves to a ball that is thrown from a height $h$
above the ground with an initial velocity
$\bm{v}_0$ at time $t=t_0$. We assume the air resistance is proportional  to the square velocity, Together with the gravitational force these are the forces acting on our system.
Note that due to the specific velocity dependence, we cannot find an analytical solution for motion in the $x$ and $y$ directions, see the discussion in Taylor after eq. (2.61).
In order to find an analytical solution we need to assume that the object is falling in the $y$-direction only. 

The position of the ball as function of time is  $\bm{r}(t)$ where $t$ is time.
 The position is measured with respect to a coordinate system with origin at the floor.

We assume we have an initial position $\bm{r}(t_0)=h\bm{e}_y$ and an initial velocity $\bm{v}_0=v_{x,0}\bm{e}_x+v_{y,0}\bm{e}_y$.

In this exercise we assume the system is influenced by the gravitational force
\[
\bm{G}=-mg\bm{e}_y
\]
and an air resistance given by a square law
\[
-Dv\bm{v}.
\]

The analytical expressions for velocity and position as functions of
time will be used to compare with the numerical results in exercise 6.

\begin{itemize}
\item 5a (3pt) Identify the forces acting on the ball and set up a diagram with the forces acting on the ball. Find the acceleration of the falling ball. 

\item 5b (4pt) Assume now that the object is falling only in the $y$-direction. Integrate the acceleration from an initial time $t_0$ to a final time $t$ and find the velocity.

\item 5c (4pt) Find thereafter the position as function of time starting with an initial time $t_0$. Find the time it takes to hit the floor.  Here you will find it convenient to set the initial velocity in the $y$-direction to zero.
\end{itemize}

\noindent
We will use the above analytical results in our numerical calculations in exercise 6. The analytical solution in the $y$-direction only will serve as a test for our numerical solution.




\paragraph{Exercise 6 (40pt), Numerical elements, solving exercise 5 numerically and adding the bouncing from the floor.}
\textbf{This exercise should be handed in as a jupyter-notebook} at D2L. Remember to write your name(s). 

Last week we:
\begin{enumerate}
\item Gained more practice with plotting in Python

\item Became familiar with arrays and representing vectors with such objects
\end{enumerate}

\noindent
This week we will:
\begin{enumerate}
\item Learn and utilize Euler's Method to find the position and the velocity

\item Compare analytical and computational solutions 

\item Add additional forces to our model
\end{enumerate}

\noindent
\begin{minted}[fontsize=\fontsize{9pt}{9pt},linenos=false,mathescape,baselinestretch=1.0,fontfamily=tt,xleftmargin=7mm]{python}
# let's start by importing useful packages we are familiar with
import numpy as np
import matplotlib.pyplot as plt
%matplotlib inline
\end{minted}

We will choose the following values
\begin{enumerate}
\item mass $m=0,2$ kg

\item accelleration (gravity) $g=9.81$ m/s$^{2}$.

\item initial position is the height $h=2$ m

\item initial velocities $v_{x,0}=v_{y,0}=10$ m/s
\end{enumerate}

\noindent
Can you find a reasonable value for the drag coefficient $D$?
You need also to define an initial time and 
the step size $\Delta t$. We can define the step size $\Delta t$ as the difference between any
two neighboring values in time (time steps) that we analyze within
some range. It can be determined by dividing the interval we are
analyzing, which in our case is time $t_{\mathrm{final}}-t_0$, by the number of steps we
are taking $(N)$. This gives us a step size $\Delta t = \dfrac{t_{\mathrm{final}}-t_0}{N}$.

With these preliminaries we are now ready to plot our results from exercise 5.

\begin{itemize}
\item 6a (10pt) Set up arrays for time, velocity, acceleration and positions for the results from exercise 5. Define an initial and final time. Choose the final time to be the time when the ball hits the ground for the first time. Make a plot of the position and velocity as functions of time.  Here you could set the initial velocity in the $y$-direction to zero and use the result from exercise 5. Else you need to try different initial times using the result from exercise 5 as a starting guess.  It is not critical if you don't reach the ground when the initial velocity in the $y$-direction is not zero.
\end{itemize}

\noindent
We move now to the numerical solution of the differential equations as discussed in the \href{{https://mhjensen.github.io/Physics321/doc/pub/motion/html/motion.html}}{lecture notes} or Malthe-Sørenssen chapter 7.5.
Let us remind ourselves about  Euler's Method.

Suppose we know $f(t)$ and its derivative $f'(t)$. To find $f(t+\Delta t)$ at the next step, $t+\Delta t$,
we can consider the Taylor expansion:

$f(t+\Delta t) = f(t) + \dfrac{(\Delta t)f'(t)}{1!} + \dfrac{(\Delta t)^2f''(t)}{2!} + ...$

If we ignore the $f''$ term and higher derivatives, we obtain

$f(t+\Delta t) \approx f(t) + (\Delta t)f'(t)$.

This approximation is the basis of Euler's method, and the Taylor
expansion suggests that it will have errors of $O(\Delta t^2)$.  Thus, one
would expect it to work better, the smaller the step size $h$ that you
use. In our case the step size is $\Delta t$. 

In setting up our code we need to

\begin{enumerate}
 \item Define and obtain all initial values, constants, and time to be analyzed with step sizes as done above (you can use the same values)

 \item Calculate the velocity using $v_{i+1} = v_{i} + (\Delta t)*a_{i}$

 \item Calculate the position using $pos_{i+1} = r_{i} + (\Delta t)*v_{i}$

 \item Calculate the new acceleration $a_{i+1}$.

 \item Repeat steps 2-4 for all time steps within a loop.

\end{enumerate}

\noindent
\item 6b (20 pt) Write a code which implements Euler's method and compute numerically and plot the position and velocity as functions of time for various values of $\Delta t$. Comment your results.



\item 6c (10pt) Compare your numerically obtained positions and velocities with the analytical results from exercise 5. In order to do this, you need to take out the motion in the $x$-direction. Comment again your results.


\paragraph{Classical Mechanics Extra Credit Assignment: Scientific Writing and attending Talks.}
The following gives you an opportunity to earn \textbf{five extra credit
points} on each of the remaining homeworks and \textbf{ten extra credit points}
on the midterms and finals.  This assignment also covers an aspect of
the scientific process that is not taught in most undergraduate
programs: scientific writing.  Writing scientific reports is how
scientist communicate their results to the rest of the field.  Knowing
how to assemble a well written scientific report will greatly benefit
you in you upper level classes, in graduate school, and in the work
place.

The full information on extra credits is found at \href{{https://github.com/mhjensen/Physics321/blob/master/doc/Homeworks/ExtraCredits/}}{\nolinkurl{https://github.com/mhjensen/Physics321/blob/master/doc/Homeworks/ExtraCredits/}}. There you will also find examples on how to write a scientific article. 
Below you can also find a description on how to gain extra credits by attending scientific talks.


This assignment allows you to gain extra credit points by practicing
your scientific writing.  For each of the remaining homeworks you can
submit the specified section of a scientific report (written about the
numerical aspect of the homework) for five extra credit points on the
assignment.  For the two midterms and the final, submitting a full
scientific report covering the numerical analysis problem will be
worth ten extra points.  For credit the grader must be able to tell
that you put effort into the assignment (i.e.~well written, well
formatted, etc.).  If you are unfamiliar with writing scientific
reports, \href{{https://github.com/mhjensen/Physics321/blob/master/doc/Homeworks/ExtraCredits/IntroductionScientificWriting.md}}{see the information here}

The following table explains what aspect of a scientific report is due
with which homework.  You can submit the assignment in any format you
like, in the same document as your homework, or in a different one.
Remember to cite any external references you use and include a
reference list.  There are no length requirements, but make sure what
you turn in is complete and through.  If you have any questions,
please contact Julie Butler at butler@frib.msu.edu.


\begin{quote}
\begin{tabular}{ccc}
\hline
\multicolumn{1}{c}{ HW/Project } & \multicolumn{1}{c}{ Due Date } & \multicolumn{1}{c}{ Extra Credit Assignment } \\
\hline
HW 3                       & 2-8                        & Abstract                   \\
HW 4                       & 2-15                       & Introduction               \\
HW 5                       & 2-22                       & Methods                    \\
HW 6                       & 3-1                        & Results and Discussion     \\
\textbf{Midterm 1}         & \textbf{3-12}              & \emph{Full Written Report} \\
HW 7                       & 3-22                       & Abstract                   \\
HW 8                       & 3-29                       & Introduction               \\
HW 9                       & 4-5                        & Results and Discussion     \\
\textbf{Midterm 2|} _4-16_ & \emph{Full Written Report} \\
HW 10                      & 4-26                       & Abstract                   \\
\textbf{Final}             & \textbf{4-30}              & \emph{Full Written Report} \\
\hline
\end{tabular}
\end{quote}

\noindent

You can also gain extra credits if you attend scientific talks.
This is described here.


\paragraph{Integrating Classwork With Research.}
This opportunity will allow you to earn up to 5 extra credit points on a Homework per week. These points can push you above 100\% or help make up for missed exercises.
In order to earn all points you must:

\begin{enumerate}
\item Attend an MSU research talk (recommended research oriented Clubs is  provided below)

\item Summarize the talk using at least 150 words

\item Turn in the summary along with your Homework.
\end{enumerate}

\noindent
Approved talks:
Talks given by researchers through the following clubs:
\begin{itemize}
\item Research and Idea Sharing Enterprise (RAISE)​: Meets Wednesday Nights Society for Physics Students (SPS)​: Meets Monday Nights

\item Astronomy Club​: Meets Monday Nights

\item Facility For Rare Isotope Beam (FRIB) Seminars: ​Occur multiple times a week
\end{itemize}

\noindent
If you have any questions please consult Jeremy Rebenstock, rebensto@msu.edu.

All the material on extra credits is at \href{{https://github.com/mhjensen/Physics321/blob/master/doc/Homeworks/ExtraCredits/}}{\nolinkurl{https://github.com/mhjensen/Physics321/blob/master/doc/Homeworks/ExtraCredits/}}. 

% ------------------- end of main content ---------------

\end{document}

