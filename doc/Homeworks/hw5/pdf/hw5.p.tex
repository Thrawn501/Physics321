%%
%% Automatically generated file from DocOnce source
%% (https://github.com/hplgit/doconce/)
%%
%%
% #ifdef PTEX2TEX_EXPLANATION
%%
%% The file follows the ptex2tex extended LaTeX format, see
%% ptex2tex: https://code.google.com/p/ptex2tex/
%%
%% Run
%%      ptex2tex myfile
%% or
%%      doconce ptex2tex myfile
%%
%% to turn myfile.p.tex into an ordinary LaTeX file myfile.tex.
%% (The ptex2tex program: https://code.google.com/p/ptex2tex)
%% Many preprocess options can be added to ptex2tex or doconce ptex2tex
%%
%%      ptex2tex -DMINTED myfile
%%      doconce ptex2tex myfile envir=minted
%%
%% ptex2tex will typeset code environments according to a global or local
%% .ptex2tex.cfg configure file. doconce ptex2tex will typeset code
%% according to options on the command line (just type doconce ptex2tex to
%% see examples). If doconce ptex2tex has envir=minted, it enables the
%% minted style without needing -DMINTED.
% #endif

% #define PREAMBLE

% #ifdef PREAMBLE
%-------------------- begin preamble ----------------------

\documentclass[%
oneside,                 % oneside: electronic viewing, twoside: printing
final,                   % draft: marks overfull hboxes, figures with paths
10pt]{article}

\listfiles               %  print all files needed to compile this document

\usepackage{relsize,makeidx,color,setspace,amsmath,amsfonts,amssymb}
\usepackage[table]{xcolor}
\usepackage{bm,ltablex,microtype}

\usepackage[pdftex]{graphicx}

\usepackage{ptex2tex}
% #ifdef MINTED
\usepackage{minted}
\usemintedstyle{default}
% #endif

\usepackage[T1]{fontenc}
%\usepackage[latin1]{inputenc}
\usepackage{ucs}
\usepackage[utf8x]{inputenc}

\usepackage{lmodern}         % Latin Modern fonts derived from Computer Modern

% Hyperlinks in PDF:
\definecolor{linkcolor}{rgb}{0,0,0.4}
\usepackage{hyperref}
\hypersetup{
    breaklinks=true,
    colorlinks=true,
    linkcolor=linkcolor,
    urlcolor=linkcolor,
    citecolor=black,
    filecolor=black,
    %filecolor=blue,
    pdfmenubar=true,
    pdftoolbar=true,
    bookmarksdepth=3   % Uncomment (and tweak) for PDF bookmarks with more levels than the TOC
    }
%\hyperbaseurl{}   % hyperlinks are relative to this root

\setcounter{tocdepth}{2}  % levels in table of contents

% prevent orhpans and widows
\clubpenalty = 10000
\widowpenalty = 10000

% --- end of standard preamble for documents ---


% insert custom LaTeX commands...

\raggedbottom
\makeindex
\usepackage[totoc]{idxlayout}   % for index in the toc
\usepackage[nottoc]{tocbibind}  % for references/bibliography in the toc

%-------------------- end preamble ----------------------

\begin{document}

% matching end for #ifdef PREAMBLE
% #endif

\newcommand{\exercisesection}[1]{\subsection*{#1}}


% ------------------- main content ----------------------



% ----------------- title -------------------------

\thispagestyle{empty}

\begin{center}
{\LARGE\bf
\begin{spacing}{1.25}
PHY321: Classical Mechanics 1
\end{spacing}
}
\end{center}

% ----------------- author(s) -------------------------

\begin{center}
{\bf Homework 5, due Monday  February 22${}^{}$} \\ [0mm]
\end{center}

\begin{center}
% List of all institutions:
\end{center}
    
% ----------------- end author(s) -------------------------

% --- begin date ---
\begin{center}
Feb 13, 2021
\end{center}
% --- end date ---

\vspace{1cm}


\paragraph{Practicalities about  homeworks and projects.}
\begin{enumerate}
\item You can work in groups (optimal groups are often 2-3 people) or by yourself. If you work as a group you can hand in one answer only if you wish. \textbf{Remember to write your name(s)}!

\item Homeworks are available ten days before the deadline.

\item How do I(we)  hand in?  You can hand in the paper and pencil exercises as a scanned document. For this homework this applies to exercises 1-5. Alternatively, you can hand in everyhting (if you are ok with typing mathematical formulae using say Latex) as a jupyter notebook at D2L. The numerical exercise(s) (exercise 6 here) should always be handed in as a jupyter notebook by the deadline at D2L. 
\end{enumerate}

\noindent
\paragraph{Introduction to homework 5.}
This week's sets of classical pen and paper and computational
exercises are a continuation of the topics from the previous homework
set and lectures from the last two weeks. We keep dealing with simple motion problems and conservation
laws; energy, momentum and angular momentum. These conservation laws
are central in Physics and understanding them properly lays the
foundation for understanding and analyzing more complicated physics
problems.

The relevant reading background is
\begin{enumerate}
\item chapters 3 and 4 of Taylor (there are many good examples there) and

\item chapters 10-14 of Malthe-Sørenssen.
\end{enumerate}

\noindent
In both textbooks there are many nice worked out examples. Malthe-Sørenssen's text contains also several coding examples you may find useful. 


The numerical homework focuses on another motion problem where you can
use the code you developed in homework 3, almost entirely. Please take
a look at the posted solution (jupyter-notebook) for homework 3. You
need only to change the forces at play. The problem at hand is a
classic, the gravitational force acting between the Sun and the
Earth. Here you will notice also that the standard Euler-integration
algorithm is not the best choice and we will introduce the so-called
Euler-Cromer method and the Velocity-Verlet method. These methods will
give much more stable numerical results with only few additions to
your code.

The code you develop here will also be reused when we analyze energy
conservation in homework set 6. And for those of you doing the honors
project, it serves as a starting point for the solar system variant.


\paragraph{Exercise 1 (15 pt), Work-energy theorem and conservation laws.}
This exercise was partly discussed during the lectures of the week of February 1-5, see URL:https://mhjensen.github.io/Physics321/doc/pub/week5/html/week5.html.

We will study a classical electron which moves in the $x$-direction along a surface. The force from the surface is
\[
\bm{F}(x)=-F_0\sin{(\frac{2\pi x}{b})}\bm{e}_x.
\]
The constant $b$ represents the distance between atoms at the surface of the material, $F_0$ is a constant and $x$ is the position of the electron.

\begin{itemize}
\item 1a (2pt) Is this a conservative force? And if so, what does that imply?

\item 1b (4pt) Use the work-energy theorem to find the velocity $v(x)$. 

\item 1c (4pt) With the above expression for the force, find the potential energy.

\item 1d (5pt) Make a plot of the potential energy and discuss the equilibrium points where the force on the electron is zero. Discuss the physical interpretation of stable and unstable equilibrium points. Use energy conservation. Recommended read here  is Malthe-Sørenssen chapter 11.3.2.
\end{itemize}

\noindent
\paragraph{Exsercise 2 (15pt), Rocket, Momentum and mass.}
Taylor exercise 3.11.   It is an exercise meant to illustrate momentum conservation or not. 
This exercise was partly discussed during the lectures, see the notes from the week of February 8-12 on \href{{https://mhjensen.github.io/Physics321/doc/pub/week6/html/week6.html}}{Energy and Momentum etc, see the part on Momentum conservation}. Taylor's chapter 3.2 covers also this example.


\paragraph{Exercise 3 (10pt), More Rockets.}
Taylor exercises 3.13 (5pt) and 3.14 (5pt). This is a continuation of
the previous exercise and most of the relevant background material can
be found in Taylor chapter 3.2.

\paragraph{Exercise 4 (10pt), Center of mass.}
Taylor exercise 3.20. Here Taylor's chapter 3.3 can be of use. This
relation will turn out to be very useful when we discuss systems of
many classical particles.




\paragraph{Exercise 5 (10pt), Warming up for the Earth-Sun system, Scaling the Equations.}
The aim of this exercise (as well as the next) is to study the motion
of objects under the influence of the gravitational force.  We will
limit ourselves to the Earth-Sun system. Here we will scale the
equations and sketch our first algorithm for solving the equations,
namely using Euler's method again, as we did in homework sets 3 and 4.  This part
together with the numerical part forms also the entry point for the
solar system honors project. Furthermore, we will reuse parts of these
results when analyzing energy conservation in homework 6.

We will limit ourselves (in order to test the algorithm) to a
hypothetical solar system with the Earth only orbiting around the sun.
The only force in the problem is gravity. Newton's law of gravitation
is given by a force $F_G$

\[
F_G=\frac{GM_{\odot}M_{\mathrm{Earth}}}{r^2},
\]

where $M_{\odot}$ is the mass of the Sun and $M_{\mathrm{Earth}}$ is
the mass of the Earth. The gravitational constant is $G$ and $r$ is
the distance between the Earth and the Sun.  The Sun
has a mass which is much larger than that of the Earth. We can
therefore safely neglect the motion of the Sun in this problem.


We assume that the orbit of the Earth around the Sun 
is co-planar, and we take this to be the $xy$-plane.
Using Newton's second law of motion we get the following equations
\[
\frac{d^2x}{dt^2}=\frac{F_{G,x}}{M_{\mathrm{Earth}}},
\]
and
\[
\frac{d^2y}{dt^2}=\frac{F_{G,y}}{M_{\mathrm{Earth}}},
\]

where $F_{G,x}$ and $F_{G,y}$ are the $x$ and $y$ components of the
gravitational force.

We will use so-called astronomical units when rewriting our equations.
Using astronomical units (AU as abbreviation)it means that one
astronomical unit of length, known as 1 AU, is the average distance
between the Sun and Earth, that is $1$ AU = $1.5\times 10^{11}$ m.  It
can also be convenient to use years instead of seconds since years
match better the time evolution of the solar system. The mass of the
Sun is $M_{\mathrm{sun}}=M_{\odot}=2\times 10^{30}$ kg. The masses of
all relevant planets and their distances from the sun are listed in
the table here in kg and AU.


\begin{quote}
\begin{tabular}{ccc}
\hline
\multicolumn{1}{c}{ Planet } & \multicolumn{1}{c}{ Mass in kg } & \multicolumn{1}{c}{ Distance to  sun in AU } \\
\hline
Earth   & $M_{\mathrm{Earth}}=6\times 10^{24}$ kg     & 1AU                    \\
Jupiter & $M_{\mathrm{Jupiter}}=1.9\times 10^{27}$ kg & 5.20 AU                \\
Mars    & $M_{\mathrm{Mars}}=6.6\times 10^{23}$ kg    & 1.52 AU                \\
Venus   & $M_{\mathrm{Venus}}=4.9\times 10^{24}$ kg   & 0.72 AU                \\
Saturn  & $M_{\mathrm{Saturn}}=5.5\times 10^{26}$ kg  & 9.54 AU                \\
Mercury & $M_{\mathrm{Mercury}}=3.3\times 10^{23}$ kg & 0.39 AU                \\
Uranus  & $M_{\mathrm{Uranus}}=8.8\times 10^{25}$ kg  & 19.19 AU               \\
Neptun  & $M_{\mathrm{Neptun}}=1.03\times 10^{26}$ kg & 30.06 AU               \\
Pluto   & $M_{\mathrm{Pluto}}=1.31\times 10^{22}$ kg  & 39.53 AU               \\
\hline
\end{tabular}
\end{quote}

\noindent

In setting up the equations we limit ourselves to a co-planar
motion and use only the $x$ and $y$ coordinates. But you should feel
free to extend your equations to three dimensions, it is not very
difficult and the data from NASA are all in three dimensions.

\href{{http://www.nasa.gov/index.html}}{NASA} has an excellent site at \href{{http://ssd.jpl.nasa.gov/horizons.cgi#top}}{\nolinkurl{http://ssd.jpl.nasa.gov/horizons.cgi\#top}}.
From there you can extract initial conditions in order to start your differential equation solver.
At the above website you need to change from \textbf{OBSERVER} to \textbf{VECTOR} and then write in the planet you are interested in.
The generated data contain the $x$, $y$ and $z$ values as well as their corresponding velocities. The velocities are in units of AU per day.
Alternatively they can be obtained in terms of km and km/s. 

For the system below involving only the Earth and the Sun, you
could just initialize the position with say $x=1$ AU and $y=0$ AU.



We assume that mass units can be obtained by using the fact that
Earth's orbit is almost circular around the Sun.

For circular motion we know that the force must obey the following relation
\[
F_G= \frac{M_{\mathrm{Earth}}v^2}{r}=\frac{GM_{\odot}M_{\mathrm{Earth}}}{r^2},
\]
where $v$ is the velocity of Earth. 
The latter equation can be used to show that
\[
v^2r=GM_{\odot}=4\pi^2\mathrm{AU}^3/\mathrm{yr}^2.
\]
\begin{itemize}
\item 5a (5pt) Show how to derive the last equation and use this to scale the differential equations, getting thus rid of the constant $G$ and the two masses. Split the differential equations for the motion in the $x$ and $y$ directions in terms of four coupled differential equations.

\item 5b (5pt)  Discretize the above differential equations and set up an algorithm for solving these equations using Euler's forward algorithm and the so-called velocity Verlet method \href{{https://mhjensen.github.io/Physics321/doc/pub/week7/html/week7.html}}{discussed in the lecture notes}. Here you can reuse what you did in homework sets 3 and 4.
\end{itemize}

\noindent
\paragraph{Exercise 6 (40pt), Numerical elements, solving exercise 5 numerically.}
\textbf{This exercise should be handed in as a jupyter-notebook} at D2L. Remember to write your name(s). 

In homework sets 3 and 4 we:
\begin{enumerate}
\item studied  Euler's Method to find the position and the velocity of a falling object, including air resistance and gravity

\item and added additional forces to our model and studied a bouncing ball.
\end{enumerate}

\noindent
This week we will reuse our code from homework 4 (exercises 6)
and replace the air resistance and force from the ground with the
gravitational force. Then we will study the stability of the system as function of initial conditions and the time step length.

We start by importing some standard packages
\bpycod
# let's start by importing useful packages we are familiar with
import numpy as np
from math import *
import matplotlib.pyplot as plt
%matplotlib inline
\epycod

\begin{itemize}
\item 6a (30 pt)  Write then a program which solves the above differential equations for the Earth-Sun system using Euler's  method and the velocity Verlet method.  Find out which initial value for the velocity that gives a circular orbit and test the stability of your algorithm as function of different time steps $\Delta t$.  Make a plot of the results you obtain for the position of the Earth (plot the $x$ and $y$ values and/or if you prefer to use three dimensions the $z$-value as well) orbiting  the Sun. Discuss eventual differences between the Verlet algorithm and the Euler algorithm. 

\item 6b (10pt) Consider then a planet which begins at a distance of 1 AU from the sun. Find out by trial and error what the initial velocity must be in order for the planet to escape from the sun.  Can you find an exact answer?  How does that match your numerical results?
\end{itemize}

\noindent
Here we add a code-example which may aid in the above studies using Euler's forward method.
\bpycod
DeltaT = 0.001
#set up arrays 
tfinal = 10 # in years
n = ceil(tfinal/DeltaT)
# set up arrays for time t, velocity v, and position r
t = np.zeros(n)
v = np.zeros((n,2))
r = np.zeros((n,2))
# Initial conditions as compact 2-dimensional arrays
r0 = np.array([1.0,0.0])
v0 = np.array([0.0,2*pi])
r[0] = r0
v[0] = v0
Fourpi2 = 4*pi*pi
# Start integrating using Euler's method
for i in range(n-1):
    # Set up the acceleration
    # Here you could have defined your own function for this
    rabs = sqrt(sum(r[i]*r[i]))
    a =  -Fourpi2*r[i]/(rabs**3)
    # update velocity, time and position using Euler's forward method
    v[i+1] = v[i] + DeltaT*a
    r[i+1] = r[i] + DeltaT*v[i]
    t[i+1] = t[i] + DeltaT
\epycod




\paragraph{Classical Mechanics Extra Credit Assignment: Scientific Writing and attending Talks.}
The following gives you an opportunity to earn \textbf{five extra credit
points} on each of the remaining homeworks and \textbf{ten extra credit points}
on the midterms and finals.  This assignment also covers an aspect of
the scientific process that is not taught in most undergraduate
programs: scientific writing.  Writing scientific reports is how
scientist communicate their results to the rest of the field.  Knowing
how to assemble a well written scientific report will greatly benefit
you in you upper level classes, in graduate school, and in the work
place.

The full information on extra credits is found at \href{{https://github.com/mhjensen/Physics321/blob/master/doc/Homeworks/ExtraCredits/}}{\nolinkurl{https://github.com/mhjensen/Physics321/blob/master/doc/Homeworks/ExtraCredits/}}. There you will also find examples on how to write a scientific article. 
Below you can also find a description on how to gain extra credits by attending scientific talks.


This assignment allows you to gain extra credit points by practicing
your scientific writing.  For each of the remaining homeworks you can
submit the specified section of a scientific report (written about the
numerical aspect of the homework) for five extra credit points on the
assignment.  For the two midterms and the final, submitting a full
scientific report covering the numerical analysis problem will be
worth ten extra points.  For credit the grader must be able to tell
that you put effort into the assignment (i.e.~well written, well
formatted, etc.).  If you are unfamiliar with writing scientific
reports, \href{{https://github.com/mhjensen/Physics321/blob/master/doc/Homeworks/ExtraCredits/IntroductionScientificWriting.md}}{see the information here}

The following table explains what aspect of a scientific report is due
with which homework.  You can submit the assignment in any format you
like, in the same document as your homework, or in a different one.
Remember to cite any external references you use and include a
reference list.  There are no length requirements, but make sure what
you turn in is complete and through.  If you have any questions,
please contact Julie Butler at butler@frib.msu.edu.


\begin{quote}
\begin{tabular}{ccc}
\hline
\multicolumn{1}{c}{ HW/Project } & \multicolumn{1}{c}{ Due Date } & \multicolumn{1}{c}{ Extra Credit Assignment } \\
\hline
HW 3                       & 2-8                        & Abstract                   \\
HW 4                       & 2-15                       & Introduction               \\
HW 5                       & 2-22                       & Methods                    \\
HW 6                       & 3-1                        & Results and Discussion     \\
\textbf{Midterm 1}         & \textbf{3-12}              & \emph{Full Written Report} \\
HW 7                       & 3-22                       & Abstract                   \\
HW 8                       & 3-29                       & Introduction               \\
HW 9                       & 4-5                        & Results and Discussion     \\
\textbf{Midterm 2|} _4-16_ & \emph{Full Written Report} \\
HW 10                      & 4-26                       & Abstract                   \\
\textbf{Final}             & \textbf{4-30}              & \emph{Full Written Report} \\
\hline
\end{tabular}
\end{quote}

\noindent

You can also gain extra credits if you attend scientific talks.
This is described here.


\paragraph{Integrating Classwork With Research.}
This opportunity will allow you to earn up to 5 extra credit points on a Homework per week. These points can push you above 100\% or help make up for missed exercises.
In order to earn all points you must:

\begin{enumerate}
\item Attend an MSU research talk (recommended research oriented Clubs is  provided below)

\item Summarize the talk using at least 150 words

\item Turn in the summary along with your Homework.
\end{enumerate}

\noindent
Approved talks:
Talks given by researchers through the following clubs:
\begin{itemize}
\item Research and Idea Sharing Enterprise (RAISE)​: Meets Wednesday Nights Society for Physics Students (SPS)​: Meets Monday Nights

\item Astronomy Club​: Meets Monday Nights

\item Facility For Rare Isotope Beam (FRIB) Seminars: ​Occur multiple times a week
\end{itemize}

\noindent
If you have any questions please consult Jeremy Rebenstock, rebensto@msu.edu.

All the material on extra credits is at \href{{https://github.com/mhjensen/Physics321/blob/master/doc/Homeworks/ExtraCredits/}}{\nolinkurl{https://github.com/mhjensen/Physics321/blob/master/doc/Homeworks/ExtraCredits/}}. 






% ------------------- end of main content ---------------

% #ifdef PREAMBLE
\end{document}
% #endif

